\section{Introduction}

Automated essay scoring has received increased attention in recent years due to the mainstream adoption of digital education platforms and the proliferation of annotated corpora on which to evaluate such systems. One emerging application of this field is automatic proficiency classification: assessing language learner text and mapping their level on a scale, e.g.\ the Common European Framework of Reference for Languages (CEFR). For a time, research had mostly centered around English, though efforts have been made to expand the landscape to other languages such as German \citep{hancke2013-german, weiss2019-german}, Swedish \citep{ostling2013-swedish, pilan2016-swedish}, Estonian \citep{vajjala2014-estonian}, Norwegian \citep{berggren2019-norwegian}, Spanish \citep{delrio2019b}, and even cross-lingually for German, Czech, and Italian \citep{vajjala2018-cefr}.

Following the introduction of a Portuguese learner corpus \citep{delrio2018}, the first automatic proficiency classification tests for the ``language of Camões" have surfaced \citep{delrio2019a, delrio2019b}. In these studies, del Río initially experimented with bag-of-words, POS and dependency \textit{n}-grams, and a modest amount of linguistic complexity features, obtaining 72\% accuracy with bag-of-words and POS \textit{n}-grams being the best performing features. She later investigated cross-lingual Spanish-Portuguese classification in view of their morphosyntactic proximity, but did not surpass the results of her previous work.

Since linguistic complexity has been identified as one of the core dimensions characterizing language proficiency \citep{housen2009}, one would expect it to be a well-suited metric for measuring growth. The lack of resources available in this area for Iberian languages, however, has heretofore impeded further research. In response, this thesis aims to expand support for extracting complexity features of European Portuguese and explore a larger set of feature interactions to determine whether it can indeed be a better indicator of language proficiency. To that end, it will provide a new platform on which to calculate such measures for Portuguese in a multilingual context.
