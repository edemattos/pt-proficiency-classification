
\pagebreak

\section{Conclusion}

This thesis has presented the first purely linguistic complexity analysis for the automatic proficiency classification of {\scshape l2} Portuguese. Until recently, Portuguese had been largely neglected in this space, though its prominence as a global language calls for its inclusion. A large and diverse set of complexity features were compiled, with several global, lexical, morphosyntactic, and discursive features demonstrating good predictive ability, corroborating the assertions of similar analyses for other languages. This thesis has shown that complexity features can indeed be leveraged to classify proficiency, though other metrics including part-of-speech \textit{n}-grams, errors, and task information may be combined for better results. In addition, resources fine-tuned for Portuguese learner text will likely improve accuracy even further.

\pagebreak