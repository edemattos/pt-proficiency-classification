
\begin{abstract}
    
This thesis explores the interaction between different linguistic complexity features and their efficacy in automatically classifying proficiency levels in {\scshape l2} Portuguese learners using supervised machine learning. We have found that even a small subset of complexity measures are useful, though supplementing them with other information may lead to further improvements. This analysis also corroborates previous findings in second language acquisition research that has posited various complexity features as good indicators of language growth.

\end{abstract}
\pagebreak